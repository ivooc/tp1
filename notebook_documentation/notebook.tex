
% Default to the notebook output style

    


% Inherit from the specified cell style.




    
\documentclass[11pt]{article}

    
    
    \usepackage[T1]{fontenc}
    % Nicer default font (+ math font) than Computer Modern for most use cases
    \usepackage{mathpazo}

    % Basic figure setup, for now with no caption control since it's done
    % automatically by Pandoc (which extracts ![](path) syntax from Markdown).
    \usepackage{graphicx}
    % We will generate all images so they have a width \maxwidth. This means
    % that they will get their normal width if they fit onto the page, but
    % are scaled down if they would overflow the margins.
    \makeatletter
    \def\maxwidth{\ifdim\Gin@nat@width>\linewidth\linewidth
    \else\Gin@nat@width\fi}
    \makeatother
    \let\Oldincludegraphics\includegraphics
    % Set max figure width to be 80% of text width, for now hardcoded.
    \renewcommand{\includegraphics}[1]{\Oldincludegraphics[width=.8\maxwidth]{#1}}
    % Ensure that by default, figures have no caption (until we provide a
    % proper Figure object with a Caption API and a way to capture that
    % in the conversion process - todo).
    \usepackage{caption}
    \DeclareCaptionLabelFormat{nolabel}{}
    \captionsetup{labelformat=nolabel}

    \usepackage{adjustbox} % Used to constrain images to a maximum size 
    \usepackage{xcolor} % Allow colors to be defined
    \usepackage{enumerate} % Needed for markdown enumerations to work
    \usepackage{geometry} % Used to adjust the document margins
    \usepackage{amsmath} % Equations
    \usepackage{amssymb} % Equations
    \usepackage{textcomp} % defines textquotesingle
    % Hack from http://tex.stackexchange.com/a/47451/13684:
    \AtBeginDocument{%
        \def\PYZsq{\textquotesingle}% Upright quotes in Pygmentized code
    }
    \usepackage{upquote} % Upright quotes for verbatim code
    \usepackage{eurosym} % defines \euro
    \usepackage[mathletters]{ucs} % Extended unicode (utf-8) support
    \usepackage[utf8x]{inputenc} % Allow utf-8 characters in the tex document
    \usepackage{fancyvrb} % verbatim replacement that allows latex
    \usepackage{grffile} % extends the file name processing of package graphics 
                         % to support a larger range 
    % The hyperref package gives us a pdf with properly built
    % internal navigation ('pdf bookmarks' for the table of contents,
    % internal cross-reference links, web links for URLs, etc.)
    \usepackage{hyperref}
    \usepackage{longtable} % longtable support required by pandoc >1.10
    \usepackage{booktabs}  % table support for pandoc > 1.12.2
    \usepackage[inline]{enumitem} % IRkernel/repr support (it uses the enumerate* environment)
    \usepackage[normalem]{ulem} % ulem is needed to support strikethroughs (\sout)
                                % normalem makes italics be italics, not underlines
    

    
    
    % Colors for the hyperref package
    \definecolor{urlcolor}{rgb}{0,.145,.698}
    \definecolor{linkcolor}{rgb}{.71,0.21,0.01}
    \definecolor{citecolor}{rgb}{.12,.54,.11}

    % ANSI colors
    \definecolor{ansi-black}{HTML}{3E424D}
    \definecolor{ansi-black-intense}{HTML}{282C36}
    \definecolor{ansi-red}{HTML}{E75C58}
    \definecolor{ansi-red-intense}{HTML}{B22B31}
    \definecolor{ansi-green}{HTML}{00A250}
    \definecolor{ansi-green-intense}{HTML}{007427}
    \definecolor{ansi-yellow}{HTML}{DDB62B}
    \definecolor{ansi-yellow-intense}{HTML}{B27D12}
    \definecolor{ansi-blue}{HTML}{208FFB}
    \definecolor{ansi-blue-intense}{HTML}{0065CA}
    \definecolor{ansi-magenta}{HTML}{D160C4}
    \definecolor{ansi-magenta-intense}{HTML}{A03196}
    \definecolor{ansi-cyan}{HTML}{60C6C8}
    \definecolor{ansi-cyan-intense}{HTML}{258F8F}
    \definecolor{ansi-white}{HTML}{C5C1B4}
    \definecolor{ansi-white-intense}{HTML}{A1A6B2}

    % commands and environments needed by pandoc snippets
    % extracted from the output of `pandoc -s`
    \providecommand{\tightlist}{%
      \setlength{\itemsep}{0pt}\setlength{\parskip}{0pt}}
    \DefineVerbatimEnvironment{Highlighting}{Verbatim}{commandchars=\\\{\}}
    % Add ',fontsize=\small' for more characters per line
    \newenvironment{Shaded}{}{}
    \newcommand{\KeywordTok}[1]{\textcolor[rgb]{0.00,0.44,0.13}{\textbf{{#1}}}}
    \newcommand{\DataTypeTok}[1]{\textcolor[rgb]{0.56,0.13,0.00}{{#1}}}
    \newcommand{\DecValTok}[1]{\textcolor[rgb]{0.25,0.63,0.44}{{#1}}}
    \newcommand{\BaseNTok}[1]{\textcolor[rgb]{0.25,0.63,0.44}{{#1}}}
    \newcommand{\FloatTok}[1]{\textcolor[rgb]{0.25,0.63,0.44}{{#1}}}
    \newcommand{\CharTok}[1]{\textcolor[rgb]{0.25,0.44,0.63}{{#1}}}
    \newcommand{\StringTok}[1]{\textcolor[rgb]{0.25,0.44,0.63}{{#1}}}
    \newcommand{\CommentTok}[1]{\textcolor[rgb]{0.38,0.63,0.69}{\textit{{#1}}}}
    \newcommand{\OtherTok}[1]{\textcolor[rgb]{0.00,0.44,0.13}{{#1}}}
    \newcommand{\AlertTok}[1]{\textcolor[rgb]{1.00,0.00,0.00}{\textbf{{#1}}}}
    \newcommand{\FunctionTok}[1]{\textcolor[rgb]{0.02,0.16,0.49}{{#1}}}
    \newcommand{\RegionMarkerTok}[1]{{#1}}
    \newcommand{\ErrorTok}[1]{\textcolor[rgb]{1.00,0.00,0.00}{\textbf{{#1}}}}
    \newcommand{\NormalTok}[1]{{#1}}
    
    % Additional commands for more recent versions of Pandoc
    \newcommand{\ConstantTok}[1]{\textcolor[rgb]{0.53,0.00,0.00}{{#1}}}
    \newcommand{\SpecialCharTok}[1]{\textcolor[rgb]{0.25,0.44,0.63}{{#1}}}
    \newcommand{\VerbatimStringTok}[1]{\textcolor[rgb]{0.25,0.44,0.63}{{#1}}}
    \newcommand{\SpecialStringTok}[1]{\textcolor[rgb]{0.73,0.40,0.53}{{#1}}}
    \newcommand{\ImportTok}[1]{{#1}}
    \newcommand{\DocumentationTok}[1]{\textcolor[rgb]{0.73,0.13,0.13}{\textit{{#1}}}}
    \newcommand{\AnnotationTok}[1]{\textcolor[rgb]{0.38,0.63,0.69}{\textbf{\textit{{#1}}}}}
    \newcommand{\CommentVarTok}[1]{\textcolor[rgb]{0.38,0.63,0.69}{\textbf{\textit{{#1}}}}}
    \newcommand{\VariableTok}[1]{\textcolor[rgb]{0.10,0.09,0.49}{{#1}}}
    \newcommand{\ControlFlowTok}[1]{\textcolor[rgb]{0.00,0.44,0.13}{\textbf{{#1}}}}
    \newcommand{\OperatorTok}[1]{\textcolor[rgb]{0.40,0.40,0.40}{{#1}}}
    \newcommand{\BuiltInTok}[1]{{#1}}
    \newcommand{\ExtensionTok}[1]{{#1}}
    \newcommand{\PreprocessorTok}[1]{\textcolor[rgb]{0.74,0.48,0.00}{{#1}}}
    \newcommand{\AttributeTok}[1]{\textcolor[rgb]{0.49,0.56,0.16}{{#1}}}
    \newcommand{\InformationTok}[1]{\textcolor[rgb]{0.38,0.63,0.69}{\textbf{\textit{{#1}}}}}
    \newcommand{\WarningTok}[1]{\textcolor[rgb]{0.38,0.63,0.69}{\textbf{\textit{{#1}}}}}
    
    
    % Define a nice break command that doesn't care if a line doesn't already
    % exist.
    \def\br{\hspace*{\fill} \\* }
    % Math Jax compatability definitions
    \def\gt{>}
    \def\lt{<}
    % Document parameters
    \title{Documentacao}
    
    
    

    % Pygments definitions
    
\makeatletter
\def\PY@reset{\let\PY@it=\relax \let\PY@bf=\relax%
    \let\PY@ul=\relax \let\PY@tc=\relax%
    \let\PY@bc=\relax \let\PY@ff=\relax}
\def\PY@tok#1{\csname PY@tok@#1\endcsname}
\def\PY@toks#1+{\ifx\relax#1\empty\else%
    \PY@tok{#1}\expandafter\PY@toks\fi}
\def\PY@do#1{\PY@bc{\PY@tc{\PY@ul{%
    \PY@it{\PY@bf{\PY@ff{#1}}}}}}}
\def\PY#1#2{\PY@reset\PY@toks#1+\relax+\PY@do{#2}}

\expandafter\def\csname PY@tok@w\endcsname{\def\PY@tc##1{\textcolor[rgb]{0.73,0.73,0.73}{##1}}}
\expandafter\def\csname PY@tok@c\endcsname{\let\PY@it=\textit\def\PY@tc##1{\textcolor[rgb]{0.25,0.50,0.50}{##1}}}
\expandafter\def\csname PY@tok@cp\endcsname{\def\PY@tc##1{\textcolor[rgb]{0.74,0.48,0.00}{##1}}}
\expandafter\def\csname PY@tok@k\endcsname{\let\PY@bf=\textbf\def\PY@tc##1{\textcolor[rgb]{0.00,0.50,0.00}{##1}}}
\expandafter\def\csname PY@tok@kp\endcsname{\def\PY@tc##1{\textcolor[rgb]{0.00,0.50,0.00}{##1}}}
\expandafter\def\csname PY@tok@kt\endcsname{\def\PY@tc##1{\textcolor[rgb]{0.69,0.00,0.25}{##1}}}
\expandafter\def\csname PY@tok@o\endcsname{\def\PY@tc##1{\textcolor[rgb]{0.40,0.40,0.40}{##1}}}
\expandafter\def\csname PY@tok@ow\endcsname{\let\PY@bf=\textbf\def\PY@tc##1{\textcolor[rgb]{0.67,0.13,1.00}{##1}}}
\expandafter\def\csname PY@tok@nb\endcsname{\def\PY@tc##1{\textcolor[rgb]{0.00,0.50,0.00}{##1}}}
\expandafter\def\csname PY@tok@nf\endcsname{\def\PY@tc##1{\textcolor[rgb]{0.00,0.00,1.00}{##1}}}
\expandafter\def\csname PY@tok@nc\endcsname{\let\PY@bf=\textbf\def\PY@tc##1{\textcolor[rgb]{0.00,0.00,1.00}{##1}}}
\expandafter\def\csname PY@tok@nn\endcsname{\let\PY@bf=\textbf\def\PY@tc##1{\textcolor[rgb]{0.00,0.00,1.00}{##1}}}
\expandafter\def\csname PY@tok@ne\endcsname{\let\PY@bf=\textbf\def\PY@tc##1{\textcolor[rgb]{0.82,0.25,0.23}{##1}}}
\expandafter\def\csname PY@tok@nv\endcsname{\def\PY@tc##1{\textcolor[rgb]{0.10,0.09,0.49}{##1}}}
\expandafter\def\csname PY@tok@no\endcsname{\def\PY@tc##1{\textcolor[rgb]{0.53,0.00,0.00}{##1}}}
\expandafter\def\csname PY@tok@nl\endcsname{\def\PY@tc##1{\textcolor[rgb]{0.63,0.63,0.00}{##1}}}
\expandafter\def\csname PY@tok@ni\endcsname{\let\PY@bf=\textbf\def\PY@tc##1{\textcolor[rgb]{0.60,0.60,0.60}{##1}}}
\expandafter\def\csname PY@tok@na\endcsname{\def\PY@tc##1{\textcolor[rgb]{0.49,0.56,0.16}{##1}}}
\expandafter\def\csname PY@tok@nt\endcsname{\let\PY@bf=\textbf\def\PY@tc##1{\textcolor[rgb]{0.00,0.50,0.00}{##1}}}
\expandafter\def\csname PY@tok@nd\endcsname{\def\PY@tc##1{\textcolor[rgb]{0.67,0.13,1.00}{##1}}}
\expandafter\def\csname PY@tok@s\endcsname{\def\PY@tc##1{\textcolor[rgb]{0.73,0.13,0.13}{##1}}}
\expandafter\def\csname PY@tok@sd\endcsname{\let\PY@it=\textit\def\PY@tc##1{\textcolor[rgb]{0.73,0.13,0.13}{##1}}}
\expandafter\def\csname PY@tok@si\endcsname{\let\PY@bf=\textbf\def\PY@tc##1{\textcolor[rgb]{0.73,0.40,0.53}{##1}}}
\expandafter\def\csname PY@tok@se\endcsname{\let\PY@bf=\textbf\def\PY@tc##1{\textcolor[rgb]{0.73,0.40,0.13}{##1}}}
\expandafter\def\csname PY@tok@sr\endcsname{\def\PY@tc##1{\textcolor[rgb]{0.73,0.40,0.53}{##1}}}
\expandafter\def\csname PY@tok@ss\endcsname{\def\PY@tc##1{\textcolor[rgb]{0.10,0.09,0.49}{##1}}}
\expandafter\def\csname PY@tok@sx\endcsname{\def\PY@tc##1{\textcolor[rgb]{0.00,0.50,0.00}{##1}}}
\expandafter\def\csname PY@tok@m\endcsname{\def\PY@tc##1{\textcolor[rgb]{0.40,0.40,0.40}{##1}}}
\expandafter\def\csname PY@tok@gh\endcsname{\let\PY@bf=\textbf\def\PY@tc##1{\textcolor[rgb]{0.00,0.00,0.50}{##1}}}
\expandafter\def\csname PY@tok@gu\endcsname{\let\PY@bf=\textbf\def\PY@tc##1{\textcolor[rgb]{0.50,0.00,0.50}{##1}}}
\expandafter\def\csname PY@tok@gd\endcsname{\def\PY@tc##1{\textcolor[rgb]{0.63,0.00,0.00}{##1}}}
\expandafter\def\csname PY@tok@gi\endcsname{\def\PY@tc##1{\textcolor[rgb]{0.00,0.63,0.00}{##1}}}
\expandafter\def\csname PY@tok@gr\endcsname{\def\PY@tc##1{\textcolor[rgb]{1.00,0.00,0.00}{##1}}}
\expandafter\def\csname PY@tok@ge\endcsname{\let\PY@it=\textit}
\expandafter\def\csname PY@tok@gs\endcsname{\let\PY@bf=\textbf}
\expandafter\def\csname PY@tok@gp\endcsname{\let\PY@bf=\textbf\def\PY@tc##1{\textcolor[rgb]{0.00,0.00,0.50}{##1}}}
\expandafter\def\csname PY@tok@go\endcsname{\def\PY@tc##1{\textcolor[rgb]{0.53,0.53,0.53}{##1}}}
\expandafter\def\csname PY@tok@gt\endcsname{\def\PY@tc##1{\textcolor[rgb]{0.00,0.27,0.87}{##1}}}
\expandafter\def\csname PY@tok@err\endcsname{\def\PY@bc##1{\setlength{\fboxsep}{0pt}\fcolorbox[rgb]{1.00,0.00,0.00}{1,1,1}{\strut ##1}}}
\expandafter\def\csname PY@tok@kc\endcsname{\let\PY@bf=\textbf\def\PY@tc##1{\textcolor[rgb]{0.00,0.50,0.00}{##1}}}
\expandafter\def\csname PY@tok@kd\endcsname{\let\PY@bf=\textbf\def\PY@tc##1{\textcolor[rgb]{0.00,0.50,0.00}{##1}}}
\expandafter\def\csname PY@tok@kn\endcsname{\let\PY@bf=\textbf\def\PY@tc##1{\textcolor[rgb]{0.00,0.50,0.00}{##1}}}
\expandafter\def\csname PY@tok@kr\endcsname{\let\PY@bf=\textbf\def\PY@tc##1{\textcolor[rgb]{0.00,0.50,0.00}{##1}}}
\expandafter\def\csname PY@tok@bp\endcsname{\def\PY@tc##1{\textcolor[rgb]{0.00,0.50,0.00}{##1}}}
\expandafter\def\csname PY@tok@fm\endcsname{\def\PY@tc##1{\textcolor[rgb]{0.00,0.00,1.00}{##1}}}
\expandafter\def\csname PY@tok@vc\endcsname{\def\PY@tc##1{\textcolor[rgb]{0.10,0.09,0.49}{##1}}}
\expandafter\def\csname PY@tok@vg\endcsname{\def\PY@tc##1{\textcolor[rgb]{0.10,0.09,0.49}{##1}}}
\expandafter\def\csname PY@tok@vi\endcsname{\def\PY@tc##1{\textcolor[rgb]{0.10,0.09,0.49}{##1}}}
\expandafter\def\csname PY@tok@vm\endcsname{\def\PY@tc##1{\textcolor[rgb]{0.10,0.09,0.49}{##1}}}
\expandafter\def\csname PY@tok@sa\endcsname{\def\PY@tc##1{\textcolor[rgb]{0.73,0.13,0.13}{##1}}}
\expandafter\def\csname PY@tok@sb\endcsname{\def\PY@tc##1{\textcolor[rgb]{0.73,0.13,0.13}{##1}}}
\expandafter\def\csname PY@tok@sc\endcsname{\def\PY@tc##1{\textcolor[rgb]{0.73,0.13,0.13}{##1}}}
\expandafter\def\csname PY@tok@dl\endcsname{\def\PY@tc##1{\textcolor[rgb]{0.73,0.13,0.13}{##1}}}
\expandafter\def\csname PY@tok@s2\endcsname{\def\PY@tc##1{\textcolor[rgb]{0.73,0.13,0.13}{##1}}}
\expandafter\def\csname PY@tok@sh\endcsname{\def\PY@tc##1{\textcolor[rgb]{0.73,0.13,0.13}{##1}}}
\expandafter\def\csname PY@tok@s1\endcsname{\def\PY@tc##1{\textcolor[rgb]{0.73,0.13,0.13}{##1}}}
\expandafter\def\csname PY@tok@mb\endcsname{\def\PY@tc##1{\textcolor[rgb]{0.40,0.40,0.40}{##1}}}
\expandafter\def\csname PY@tok@mf\endcsname{\def\PY@tc##1{\textcolor[rgb]{0.40,0.40,0.40}{##1}}}
\expandafter\def\csname PY@tok@mh\endcsname{\def\PY@tc##1{\textcolor[rgb]{0.40,0.40,0.40}{##1}}}
\expandafter\def\csname PY@tok@mi\endcsname{\def\PY@tc##1{\textcolor[rgb]{0.40,0.40,0.40}{##1}}}
\expandafter\def\csname PY@tok@il\endcsname{\def\PY@tc##1{\textcolor[rgb]{0.40,0.40,0.40}{##1}}}
\expandafter\def\csname PY@tok@mo\endcsname{\def\PY@tc##1{\textcolor[rgb]{0.40,0.40,0.40}{##1}}}
\expandafter\def\csname PY@tok@ch\endcsname{\let\PY@it=\textit\def\PY@tc##1{\textcolor[rgb]{0.25,0.50,0.50}{##1}}}
\expandafter\def\csname PY@tok@cm\endcsname{\let\PY@it=\textit\def\PY@tc##1{\textcolor[rgb]{0.25,0.50,0.50}{##1}}}
\expandafter\def\csname PY@tok@cpf\endcsname{\let\PY@it=\textit\def\PY@tc##1{\textcolor[rgb]{0.25,0.50,0.50}{##1}}}
\expandafter\def\csname PY@tok@c1\endcsname{\let\PY@it=\textit\def\PY@tc##1{\textcolor[rgb]{0.25,0.50,0.50}{##1}}}
\expandafter\def\csname PY@tok@cs\endcsname{\let\PY@it=\textit\def\PY@tc##1{\textcolor[rgb]{0.25,0.50,0.50}{##1}}}

\def\PYZbs{\char`\\}
\def\PYZus{\char`\_}
\def\PYZob{\char`\{}
\def\PYZcb{\char`\}}
\def\PYZca{\char`\^}
\def\PYZam{\char`\&}
\def\PYZlt{\char`\<}
\def\PYZgt{\char`\>}
\def\PYZsh{\char`\#}
\def\PYZpc{\char`\%}
\def\PYZdl{\char`\$}
\def\PYZhy{\char`\-}
\def\PYZsq{\char`\'}
\def\PYZdq{\char`\"}
\def\PYZti{\char`\~}
% for compatibility with earlier versions
\def\PYZat{@}
\def\PYZlb{[}
\def\PYZrb{]}
\makeatother


    % Exact colors from NB
    \definecolor{incolor}{rgb}{0.0, 0.0, 0.5}
    \definecolor{outcolor}{rgb}{0.545, 0.0, 0.0}



    
    % Prevent overflowing lines due to hard-to-break entities
    \sloppy 
    % Setup hyperref package
    \hypersetup{
      breaklinks=true,  % so long urls are correctly broken across lines
      colorlinks=true,
      urlcolor=urlcolor,
      linkcolor=linkcolor,
      citecolor=citecolor,
      }
    % Slightly bigger margins than the latex defaults
    
    \geometry{verbose,tmargin=1in,bmargin=1in,lmargin=1in,rmargin=1in}
    
    

    \begin{document}
    
    
    \maketitle
    
    

    
    \section{Programação Orientada à
Objetos:}\label{programauxe7uxe3o-orientada-uxe0-objetos}

\subsection{Trabalho Prático 1: Classe
Matriz}\label{trabalho-pruxe1tico-1-classe-matriz}

Autores: Ivo Capanema, Gabriel Becker

    \subsection{Introdução:}\label{introduuxe7uxe3o}

\subsection{Como compilar e executar:}\label{como-compilar-e-executar}

Nesta parte, recomenda-se que seja abordada a forma de implementação do
código (qual sistema operacional utilizado, compilador, IDE, etc..)
Implementação do Código: \#\#\# Header le: Fazer uma breve descrição do
arquivo de cabeçalho com a implementação da classe, seus atributos e
métodos. Pode-se colocar a parte do código se julgar necessário,
conforme mostrado abaixo: \#\#\# Matrix.h

    \begin{Verbatim}[commandchars=\\\{\}]
{\color{incolor}In [{\color{incolor} }]:} \PY{c+c1}{\PYZsh{}ifndef \PYZus{}TPPOO\PYZus{}MATRIZ\PYZus{}\PYZus{}}
        \PY{c+c1}{\PYZsh{}define \PYZus{}TPPOO\PYZus{}MATRIZ\PYZus{}\PYZus{}}
        
        \PY{c+c1}{\PYZsh{}include \PYZlt{}iostream\PYZgt{}}
        \PY{c+c1}{\PYZsh{}include \PYZlt{}iomanip\PYZgt{}}
        
        \PY{n}{using} \PY{n}{namespace} \PY{n}{std}\PY{p}{;}
        
        \PY{k}{class} \PY{n+nc}{Matrix}\PY{p}{\PYZob{}}
          \PY{n+nb}{int} \PY{n}{rows}\PY{p}{;}  \PY{o}{/}\PY{o}{/} \PY{n}{numero} \PY{n}{de} \PY{n}{linhas}
          \PY{n+nb}{int} \PY{n}{cols}\PY{p}{;}  \PY{o}{/}\PY{o}{/} \PY{n}{numero} \PY{n}{de} \PY{n}{colunas}
          \PY{n}{double}\PY{o}{*} \PY{n}{pos}\PY{p}{;}  \PY{o}{/}\PY{o}{/} \PY{n}{vetor} \PY{n}{unidimensional} \PY{n}{com} \PY{n}{valores} \PY{n}{da} \PY{n}{matriz}
        \PY{p}{\PYZcb{}}\PY{p}{;}
        
        \PY{c+c1}{\PYZsh{}endif}
\end{Verbatim}


    \subsubsection{Construtores:}\label{construtores}

    \begin{Verbatim}[commandchars=\\\{\}]
{\color{incolor}In [{\color{incolor}2}]:} \PY{n+nb}{print}\PY{p}{(}\PY{l+s+s2}{\PYZdq{}}\PY{l+s+s2}{COCO}\PY{l+s+s2}{\PYZdq{}}\PY{p}{)}
\end{Verbatim}


    \begin{Verbatim}[commandchars=\\\{\}]
COCO

    \end{Verbatim}

    \begin{Verbatim}[commandchars=\\\{\}]
{\color{incolor}In [{\color{incolor}1}]:} \PY{n}{Matrix}\PY{p}{(}\PY{n+nb}{int} \PY{o}{=} \PY{l+m+mi}{0}\PY{p}{,} \PY{n+nb}{int} \PY{o}{=} \PY{l+m+mi}{0}\PY{p}{,} \PY{n}{const} \PY{n}{double}\PY{o}{\PYZam{}} \PY{o}{=} \PY{l+m+mf}{0.0}\PY{p}{)}\PY{p}{;} \PY{o}{/}\PY{o}{/} \PY{n}{construtor} \PY{n}{generico} \PY{n}{default}
        \PY{n}{Matrix}\PY{p}{(}\PY{n}{const} \PY{n}{Matrix}\PY{o}{\PYZam{}}\PY{p}{)}\PY{p}{;} \PY{o}{/}\PY{o}{/} \PY{n}{construtor} \PY{n}{de} \PY{n}{copia}
\end{Verbatim}


    \begin{Verbatim}[commandchars=\\\{\}]

          File "<ipython-input-1-c805950a71f3>", line 2
        Matrix(int = 0, int = 0, const double\& = 0.0); // construtor generico default
                                            \^{}
    SyntaxError: invalid syntax


    \end{Verbatim}

    \subsubsection{Sobrecarga de
Operadores:}\label{sobrecarga-de-operadores}

    \begin{Verbatim}[commandchars=\\\{\}]
{\color{incolor}In [{\color{incolor} }]:} \PY{n}{Matrix}\PY{o}{\PYZam{}} \PY{n}{operator}\PY{o}{=}\PY{p}{(}\PY{n}{const} \PY{n}{Matrix}\PY{o}{\PYZam{}}\PY{p}{)}\PY{p}{;} \PY{o}{/}\PY{o}{/} \PY{n}{operador} \PY{n}{de} \PY{n}{copia}
        \PY{n}{double}\PY{o}{\PYZam{}} \PY{n}{operator}\PY{p}{(}\PY{p}{)}\PY{p}{(}\PY{n}{const} \PY{n+nb}{int}\PY{o}{\PYZam{}}\PY{p}{,}\PY{n}{const} \PY{n+nb}{int}\PY{o}{\PYZam{}}\PY{p}{)}\PY{p}{;} \PY{o}{/}\PY{o}{/} \PY{n}{retorna} \PY{n}{elemento} \PY{n}{pos}\PY{p}{[}\PY{n}{x}\PY{p}{]}\PY{p}{[}\PY{n}{y}\PY{p}{]}
        \PY{n}{const} \PY{n}{Matrix}  \PY{n}{operator}\PY{o}{+} \PY{p}{(}\PY{n}{const} \PY{n}{Matrix}\PY{o}{\PYZam{}}\PY{p}{)} \PY{n}{const}\PY{p}{;}  \PY{o}{/}\PY{o}{/} \PY{n}{retorna} \PY{n}{soma} \PY{n}{de} \PY{n}{matrizes}
        \PY{n}{const} \PY{n}{Matrix}  \PY{n}{operator}\PY{o}{\PYZhy{}} \PY{p}{(}\PY{n}{const} \PY{n}{Matrix}\PY{o}{\PYZam{}}\PY{p}{)} \PY{n}{const}\PY{p}{;}  \PY{o}{/}\PY{o}{/} \PY{n}{retorna} \PY{n}{subtrai} \PY{n}{matirzes}
        \PY{n}{Matrix}\PY{o}{\PYZam{}} \PY{n}{operator}\PY{o}{+}\PY{o}{=}\PY{p}{(}\PY{n}{const} \PY{n}{Matrix}\PY{o}{\PYZam{}}\PY{p}{)}\PY{p}{;}  \PY{o}{/}\PY{o}{/} \PY{n}{soma} \PY{n}{matriz} \PY{n}{em} \PY{n}{si} \PY{n}{mesma}
        \PY{n}{Matrix}\PY{o}{\PYZam{}} \PY{n}{operator}\PY{o}{\PYZhy{}}\PY{o}{=}\PY{p}{(}\PY{n}{const} \PY{n}{Matrix}\PY{o}{\PYZam{}}\PY{p}{)}\PY{p}{;}  \PY{o}{/}\PY{o}{/} \PY{n}{subtrai} \PY{n}{matriz} \PY{n}{em} \PY{n}{si} \PY{n}{mesma}
        \PY{n}{Matrix}\PY{o}{\PYZam{}} \PY{n}{operator}\PY{o}{\PYZti{}}\PY{p}{(}\PY{p}{)} \PY{p}{;}  \PY{o}{/}\PY{o}{/} \PY{n}{transp}\PY{err}{�}\PY{n}{e} \PY{n}{matriz}
        \PY{n}{Matrix}\PY{o}{\PYZam{}} \PY{n}{operator}\PY{o}{*}\PY{o}{=}\PY{p}{(}\PY{n}{const} \PY{n}{double}\PY{o}{\PYZam{}}\PY{p}{)}\PY{p}{;}  \PY{o}{/}\PY{o}{/} \PY{n}{multiplicacao} \PY{n}{escalar} \PY{n}{da} \PY{n}{propria} \PY{n}{matriz} \PY{n}{em} \PY{n}{si} \PY{n}{mesma}
        \PY{n}{Matrix}\PY{o}{\PYZam{}} \PY{n}{operator}\PY{o}{*}\PY{o}{=}\PY{p}{(}\PY{n}{const} \PY{n}{Matrix}\PY{o}{\PYZam{}}\PY{p}{)}\PY{p}{;}  \PY{o}{/}\PY{o}{/} \PY{n}{multiplicacao} \PY{n}{vetorial} \PY{n}{de} \PY{n}{matrizes} \PY{n}{em} \PY{n}{si} \PY{n}{mesma}
        \PY{n}{const} \PY{n}{Matrix}  \PY{n}{operator}\PY{o}{*} \PY{p}{(}\PY{n}{const} \PY{n}{Matrix}\PY{o}{\PYZam{}}\PY{p}{)} \PY{n}{const}\PY{p}{;}  \PY{o}{/}\PY{o}{/} \PY{n}{multiplicacao}
        \PY{n+nb}{bool}    \PY{n}{operator}\PY{o}{==}\PY{p}{(}\PY{n}{const} \PY{n}{Matrix}\PY{o}{\PYZam{}}\PY{p}{)}\PY{p}{;}  \PY{o}{/}\PY{o}{/} \PY{n}{comparador} \PY{n}{de} \PY{n}{igualdade} \PY{n}{de} \PY{n}{matrizes}
        \PY{n+nb}{bool}    \PY{n}{operator}\PY{o}{!=}\PY{p}{(}\PY{n}{const} \PY{n}{Matrix}\PY{o}{\PYZam{}}\PY{p}{)}\PY{p}{;}  \PY{o}{/}\PY{o}{/} \PY{n}{comparador} \PY{n}{de} \PY{n}{diferenca} \PY{n}{de} \PY{n}{matrizes}
        \PY{n}{friend} \PY{n}{ostream}\PY{o}{\PYZam{}} \PY{n}{operator}\PY{o}{\PYZlt{}\PYZlt{}} \PY{p}{(}\PY{n}{ostream}\PY{o}{\PYZam{}}\PY{p}{,} \PY{n}{const} \PY{n}{Matrix}\PY{o}{\PYZam{}}\PY{p}{)}\PY{p}{;}  \PY{o}{/}\PY{o}{/} \PY{n}{imprime} \PY{n}{a} \PY{n}{matriz}
        \PY{n}{friend} \PY{n}{istream}\PY{o}{\PYZam{}} \PY{n}{operator}\PY{o}{\PYZgt{}\PYZgt{}} \PY{p}{(}\PY{n}{istream}\PY{o}{\PYZam{}}\PY{p}{,} \PY{n}{Matrix}\PY{o}{\PYZam{}}\PY{p}{)}\PY{p}{;}  \PY{o}{/}\PY{o}{/} \PY{n}{entra} \PY{n}{a} \PY{n}{matriz}
\end{Verbatim}


    \subsubsection{Getters e Setters
básicos:}\label{getters-e-setters-buxe1sicos}

    \begin{Verbatim}[commandchars=\\\{\}]
{\color{incolor}In [{\color{incolor} }]:} \PY{n+nb}{int} \PY{n}{getIndexCol} \PY{p}{(}\PY{n+nb}{int} \PY{n}{index}\PY{p}{)} \PY{n}{const} \PY{p}{\PYZob{}} \PY{k}{return} \PY{n}{index} \PY{o}{\PYZpc{}} \PY{n}{cols}\PY{p}{;} \PY{p}{\PYZcb{}}
        \PY{n+nb}{int} \PY{n}{getIndexRow} \PY{p}{(}\PY{n+nb}{int} \PY{n}{index}\PY{p}{)} \PY{n}{const} \PY{p}{\PYZob{}} \PY{k}{return} \PY{n}{index} \PY{o}{/} \PY{n}{cols}\PY{p}{;} \PY{p}{\PYZcb{}}
        \PY{n}{double} \PY{n}{getValueAt}\PY{p}{(}\PY{n+nb}{int} \PY{n}{row}\PY{p}{,} \PY{n+nb}{int} \PY{n}{col}\PY{p}{)} \PY{n}{const} \PY{p}{\PYZob{}} \PY{k}{return} \PY{n}{pos}\PY{p}{[}\PY{n}{index}\PY{p}{(}\PY{n}{row}\PY{p}{,} \PY{n}{col}\PY{p}{)}\PY{p}{]}\PY{p}{;} \PY{p}{\PYZcb{}}
        \PY{n}{void} \PY{n}{setValueAt}\PY{p}{(}\PY{n+nb}{int} \PY{n}{row}\PY{p}{,} \PY{n+nb}{int} \PY{n}{col}\PY{p}{,} \PY{n}{const} \PY{n}{double} \PY{o}{\PYZam{}}\PY{n}{value}\PY{p}{)}\PY{p}{;}
        \PY{n}{void} \PY{n}{setValueAt}\PY{p}{(}\PY{n+nb}{int} \PY{n}{index}\PY{p}{,} \PY{n}{const} \PY{n}{double} \PY{o}{\PYZam{}}\PY{n}{value}\PY{p}{)}\PY{p}{;}
        \PY{n}{Matrix}\PY{p}{(}\PY{n+nb}{int} \PY{o}{=} \PY{l+m+mi}{0}\PY{p}{,} \PY{n+nb}{int} \PY{o}{=} \PY{l+m+mi}{0}\PY{p}{,} \PY{n}{const} \PY{n}{double}\PY{o}{\PYZam{}} \PY{o}{=} \PY{l+m+mf}{0.0}\PY{p}{)}\PY{p}{;} \PY{o}{/}\PY{o}{/} \PY{n}{construtor} \PY{n}{generico} \PY{n}{default}
        \PY{n}{Matrix}\PY{p}{(}\PY{n}{const} \PY{n}{Matrix}\PY{o}{\PYZam{}}\PY{p}{)}\PY{p}{;} \PY{o}{/}\PY{o}{/} \PY{n}{construtor} \PY{n}{de} \PY{n}{copia}
        \PY{n+nb}{int} \PY{n}{getRows}\PY{p}{(}\PY{p}{)} \PY{n}{const} \PY{p}{\PYZob{}} \PY{k}{return} \PY{n}{rows}\PY{p}{;} \PY{p}{\PYZcb{}} \PY{o}{/}\PY{o}{/} \PY{n}{devolve} \PY{n}{numero} \PY{n}{de} \PY{n}{linhas}
        \PY{n+nb}{int} \PY{n}{getCols}\PY{p}{(}\PY{p}{)} \PY{n}{const} \PY{p}{\PYZob{}} \PY{k}{return} \PY{n}{cols}\PY{p}{;} \PY{p}{\PYZcb{}} \PY{o}{/}\PY{o}{/} \PY{n}{devolve} \PY{n}{numero} \PY{n}{de} \PY{n}{colunas}
        \PY{n}{double}\PY{o}{\PYZam{}} \PY{n}{operator}\PY{p}{(}\PY{p}{)}\PY{p}{(}\PY{n}{const} \PY{n+nb}{int}\PY{o}{\PYZam{}}\PY{p}{,}\PY{n}{const} \PY{n+nb}{int}\PY{o}{\PYZam{}}\PY{p}{)}\PY{p}{;} \PY{o}{/}\PY{o}{/} \PY{n}{retorna} \PY{n}{elemento} \PY{n}{pos}\PY{p}{[}\PY{n}{x}\PY{p}{]}\PY{p}{[}\PY{n}{y}\PY{p}{]}
\end{Verbatim}


    \subsubsection{Tratamento de exceções (caso
aplicável)}\label{tratamento-de-exceuxe7uxf5es-caso-aplicuxe1vel}

Falar sobre o comportamento do código, caso aplicável, quando ocorrem
operações inapropriadas (multiplicação de matrizes de dimensões
incompatíveis por exemplo) \#\#\# Main Falar sobre a função principal
implementada \#\#\# Demais tópicos: Caso houver algum tópico não
tratado, pode ser inserido aqui (lembrem-se que esse template é só uma
recomendação e que pode ser adequado à cada trabalho) \#\# Conclusão
(caso aplicável):

    \begin{Verbatim}[commandchars=\\\{\}]
{\color{incolor}In [{\color{incolor} }]:} 
\end{Verbatim}



    % Add a bibliography block to the postdoc
    
    
    
    \end{document}
